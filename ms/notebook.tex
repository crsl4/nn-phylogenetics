\documentclass[]{article}
\usepackage{lmodern}
\usepackage{amssymb,amsmath}
\usepackage{ifxetex,ifluatex}
\usepackage{fixltx2e} % provides \textsubscript
\ifnum 0\ifxetex 1\fi\ifluatex 1\fi=0 % if pdftex
  \usepackage[T1]{fontenc}
  \usepackage[utf8]{inputenc}
\else % if luatex or xelatex
  \ifxetex
    \usepackage{mathspec}
  \else
    \usepackage{fontspec}
  \fi
  \defaultfontfeatures{Ligatures=TeX,Scale=MatchLowercase}
\fi
% use upquote if available, for straight quotes in verbatim environments
\IfFileExists{upquote.sty}{\usepackage{upquote}}{}
% use microtype if available
\IfFileExists{microtype.sty}{%
\usepackage{microtype}
\UseMicrotypeSet[protrusion]{basicmath} % disable protrusion for tt fonts
}{}
\usepackage{hyperref}
\hypersetup{unicode=true,
            pdftitle={Neural Networks in Phylogenetic Inference},
            pdfauthor={Claudia Solis-Lemus, Leonardo Zepeda-Nunez},
            pdfborder={0 0 0},
            breaklinks=true}
\urlstyle{same}  % don't use monospace font for urls
\usepackage{natbib}
\bibliographystyle{apalike}
\usepackage{graphicx,grffile}
\makeatletter
\def\maxwidth{\ifdim\Gin@nat@width>\linewidth\linewidth\else\Gin@nat@width\fi}
\def\maxheight{\ifdim\Gin@nat@height>\textheight\textheight\else\Gin@nat@height\fi}
\makeatother
% Scale images if necessary, so that they will not overflow the page
% margins by default, and it is still possible to overwrite the defaults
% using explicit options in \includegraphics[width, height, ...]{}
\setkeys{Gin}{width=\maxwidth,height=\maxheight,keepaspectratio}
\IfFileExists{parskip.sty}{%
\usepackage{parskip}
}{% else
\setlength{\parindent}{0pt}
\setlength{\parskip}{6pt plus 2pt minus 1pt}
}
\setlength{\emergencystretch}{3em}  % prevent overfull lines
\providecommand{\tightlist}{%
  \setlength{\itemsep}{0pt}\setlength{\parskip}{0pt}}
\setcounter{secnumdepth}{0}
% Redefines (sub)paragraphs to behave more like sections
\ifx\paragraph\undefined\else
\let\oldparagraph\paragraph
\renewcommand{\paragraph}[1]{\oldparagraph{#1}\mbox{}}
\fi
\ifx\subparagraph\undefined\else
\let\oldsubparagraph\subparagraph
\renewcommand{\subparagraph}[1]{\oldsubparagraph{#1}\mbox{}}
\fi

%%% Use protect on footnotes to avoid problems with footnotes in titles
\let\rmarkdownfootnote\footnote%
\def\footnote{\protect\rmarkdownfootnote}

%%% Change title format to be more compact
\usepackage{titling}

% Create subtitle command for use in maketitle
\providecommand{\subtitle}[1]{
  \posttitle{
    \begin{center}\large#1\end{center}
    }
}

\setlength{\droptitle}{-2em}

  \title{Neural Networks in Phylogenetic Inference}
    \pretitle{\vspace{\droptitle}\centering\huge}
  \posttitle{\par}
    \author{Claudia Solis-Lemus, Leonardo Zepeda-Nunez}
    \preauthor{\centering\large\emph}
  \postauthor{\par}
    \date{}
    \predate{}\postdate{}
  
\usepackage[margin=1in]{geometry}
\usepackage[ruled,vlined]{algorithm2e}
\usepackage[dvipsnames]{xcolor}
\usepackage{float}
\newcommand{\missing}[1]{\textcolor{red}{\textbf{#1}}}
\usepackage{graphicx}

\begin{document}
\maketitle

\textbf{Background:} Phylogenetic inference consists of the estimation
of a bifurcating tree structure to represent the evolutionary history of
some species under study. Bayesian methods are among the most widely
used due to their ability to integrate different data sources, and their
natural representation of uncertainty through posterior distributions.
However, the estimation of these posterior distributions is
computationally heavy, because in order to estimate the posterior
distribution, we need to traverse the high-dimensional parameter space
with MCMC.

\textbf{Idea:} We can utilize neural networks to estimate the posterior
distributions. Instead of traversing the high-dimensional parameter
space, we can \emph{intelligently} choose points in the parameter space,
and then interpolate with the neural network.

\hypertarget{data}{%
\section{Data}\label{data}}

\begin{itemize}
\tightlist
\item
  \(n\) species under study
\item
  Matrix \(D \in \{A,G,C,T\}^{n \times \ell}\) with genetic sequences of
  length \(\ell\). Normally, many of the sites (columns) will be
  invariant (identical across species), but we can assume that we
  already removed invariant sites.
\end{itemize}

\hypertarget{parameters-of-interest}{%
\section{Parameters of interest}\label{parameters-of-interest}}

\begin{itemize}
\tightlist
\item
  \(T \in \mathcal{T}_n\): Bifurcating tree with \(n\) leaves.
  \(\mathcal{T}_n\) represents the space of all bifurcating trees with
  \(n\) leaves
\item
  \(Q \in \mathbf{R}^{4 \times 4}\): Transition rate matrix
\item
  \(t \in [0,\infty)^{2n-2}\): Vector of branch lengths
\end{itemize}

Thus, the parameter space is
\(\theta = (T,Q,t) \in \Theta = \mathcal{T}_n \times \mathbf{R}^{4 \times 4} \times [0,\infty)^{2n-2}\).

\hypertarget{model}{%
\section{Model}\label{model}}

\hypertarget{model-of-evolution}{%
\subsection{1. Model of evolution}\label{model-of-evolution}}

The model of evolution is a homogeneous continuous-time Markov model on
4 states: \(\{A,C,G,T\}\) along the branches of the phylogenetic tree
(see figure \ref{phylo-inf}). The model has three parameters:
\(\theta = (T,Q,t)\).

\begin{figure}[ht]
\centering
\includegraphics[width=0.5\textwidth]{figures/phylo-inference.pdf}
\caption{Phylogenetic likelihood depends on a continuous-time Markov model with 4 states, and 3 parameters: tree, $Q$ matrix and $t$ vector of branch lenghts}
\label{phylo-inf}
\end{figure}

There are different types of model of evolution that can be chosen,
depending on the assumptions we make on the \(Q\) matrix, from the
simplest (Jukes-Cantor model) to the most complex (GTR model).

Thus, as in figure \ref{phylo-inf}, the likelihood function has three
parameters (plus the data): \(L(\theta|D)\) and it depends on the model
of evolution chosen.

\hypertarget{prior-distributions}{%
\subsection{2. Prior distributions}\label{prior-distributions}}

Each parameter \((T,Q,t)\) will have a prior distribution, representing
the biological knowledge that we have on these parameters prior to
estimation with data:

\begin{itemize}
\tightlist
\item
  \(\pi_T(T)\): tree prior, usual choices are Yule or coalescent model
\item
  \(\pi_Q(Q)\): prior for \(Q\) matrix, usual choice is Dirichlet
\item
  \(\pi_t(t)\): prior for branch length, usual choice is Gamma
\end{itemize}

Thus, the prior of three parameters is: \[
\pi(\theta) = \pi_T(T) \pi_Q(Q) \pi_t(t)
\]

If no information is known, uninformative priors can be selected:
\(\pi(\theta) \propto 1\).

\hypertarget{posterior-distribution}{%
\section{Posterior distribution}\label{posterior-distribution}}

The \textbf{goal} of bayesian phylogenetic inference is to estimate the
posterior distribution of the three parameters: \[
P(\theta | D) \propto L(\theta|D) \pi(\theta)
\]

\hypertarget{current-approach}{%
\subsection{Current approach}\label{current-approach}}

Run a MCMC in parameter space \(\Theta\) for millions of iterations, and
use the evaluated posterior distribution on the visited states to
estimate the function surface (like an histogram). This approach
requires long chains to traverse the parameter space (\(\Theta\)) which
are both inefficient and time-consuming.

\hypertarget{our-proposed-idea}{%
\subsection{Our proposed idea}\label{our-proposed-idea}}

\begin{itemize}
\tightlist
\item
  Randomly (or \emph{intelligently}) select points
  \(\{\theta_i = (T_i,Q_i,t_i)\}_{i=1}^K\) in the parameter space
  \(\Theta\)
\item
  Evaluate the posterior distribution on this points
  \(\{P(\theta_i|D)\}_{i=1}^K\)
\item
  Train a neural network with these points
  \(\{(\theta_i, P(\theta_i|D))\}_{i=1}^K\)
\item
  Use neural network to interpolate the posterior distribution on
  \(\Theta\)
\end{itemize}

\bibliography{ms.bib}


\end{document}
